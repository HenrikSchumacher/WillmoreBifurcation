\section{Questions}

\subsection{\Andy{I do not have any intuition about preservation ‘up to first order’.}}

\Henrik{%
Through a somewhat intransparent process (in the sense that I did not present their derivation in the \emph{Mathematica} notebook), I was able to determine the fundamental vector fields $\Funda[][\AmbSpace]$ for the three ``infinitesimal boosts'' $u$, $v$, $w \in \Lie(G)$, i.e. for three further generators that complement $\Lie(\SO(\AmbDim)) \oplus \Lie(\AmbSpace) \oplus \Lie(\R_{>0}) in \Lie(G)$, so that we obtain a full basis of $\Lie(G)$. 
In the notebook, I denoted these three fundamental vector fields by $U$, $V$, $W$, i.e., 
\begin{align*}
	U \ceq K^{\AmbSpace}_u,
	\quad
	V \ceq K^{\AmbSpace}_v,
	\qand
	W \ceq K^{\AmbSpace}_w.
\end{align*}
Thus by \eqref{eq:Funda}, we have 
\begin{align*}
	\Funda[u] \at_f =  U \circ f, 
	\quad
	\Funda[v] \at_f =  V \circ f, 
	\qand
	\Funda[w] \at_f =  W \circ f.
\end{align*}

We can now study area $\Area \colon \ConfSpace \to \R$ and enclosed volume $\Volume \colon \ConfSpace \to \R$ along an orbit $\Orbit \ceq (G f)$ of $G$. 
By the chain rule we have for each differentiable curve 
$g \colon ]-\epsilon,\epsilon[ \to G$ with $g(0) = 1$ and $g'(0) \in \Lie(G)$ 
that
\begin{align*}
		\frac{\dd}{\dd t} \Area( L_{g(t)}(f) ) \At_{t=0} = D \Area(f) \,(\Funda[\xi] \at_f)
\end{align*}
and analogously
\begin{align*}
		\frac{\dd}{\dd t} \Volume( L_{g(t)}(f) ) \At_{t=0} = D \Volume(f) \,(\Funda[\xi] \at_f).
\end{align*}

And what I did in the notebook is just checking for each torus $f$ of revolution with circular contour (TRCC for short) that

\begin{align*}
	D \Area(f) \,(\Funda[u] \at_f) &= 0.
	\\
	D \Area(f) \,(\Funda[v] \at_f) &= 0,
	\\
	D \Area(f) \,(\Funda[w] \at_f) &= 0,
	\\
	D \Volume(f) \,(\Funda[u] \at_f) &= 0,
	\\
	D \Volume(f) \,(\Funda[v] \at_f) &= 0,
	\\
	D \Volume(f) \,(\Funda[w] \at_f) &= 0.
\end{align*}

Of course, this does not necessarily imply that $\Area$ and $\Volume$ are constant on the whole orbit $\mathcal{O} = (G f)$, not even on an arbitrarily small neighborhood of $f$. While the boost $\exp(t \, w)$ ``in $z$-direction'' (along the axis of revolution) sends a TRCC to a TRCC (the contour has to be circular again and boosting along the axis of revolution preserves rotation symmetry; the parameterization might change considerably, but we are not interested in the parameterization, right?), the boosts $\exp(t \, u)$ and $\exp(t \, v )$ along the $x$ and $y$ direction should deform TRCC into Dupin cyclides. And so I do not expect that $\exp(t \,u)$, $\exp(t \,v)$ preserve both area and volume. But I could be wrong.
}%

\subsection{\Andy{With your symbolic setup in \emph{Mathematica} Henrik, is it easy to look at the 2nd order change? Maybe we will even see what makes the Clifford torus isoperimetric ratio so special?}}

\Henrik{%
Could be worth a try. Computing the Hessian at a critical point is typically not that difficult because it actually does not depend on the employed Riemannian metric and its connection:

Denote the orbit of $f$ under $G$ by $\Orbit = L_G(f)$. The fundamental vector fields $\Funda$ span the tangent space  $T_f  \Orbit$. If $\Function \colon \Orbit \to \R$ is twice differentiable, then for any Riemannian metric $h$ on $\Orbit$ we have
\begin{align*}
 	\Hess^{h}(\Function)( \Funda[\xi] , \Funda[\eta]) 
 	= 
 	\bigparen{ \Funda[\xi] \Funda[\eta] \Function } \at_f 
 	- 
 	\dd \Function(f) \, \nabla^h_{\Funda[\xi]} \Funda[\eta],
 	\quad
 	\text{for all $\xi$, $\eta \in G$,}
\end{align*} 	
where $\nabla^h$ is the Levi-Civita connection of $h$.
So in general, the result depends on the choice of a metric! But if $\dd \Function \at f = 0$ (like in the cases $\Function = \Area |_{\Orbit}$ and $\Function = \Volume |_{\Orbit}$ for example), then the metric-dependent term just vanishes and we get
\begin{align*}
 	\Hess^{h}(\Function)( \Funda[\xi] , \Funda[\eta]) 
 	= 
 	\bigparen{ \Funda[\xi] \Funda[\eta] \Function } \at_f 
	.\textsl{•}
\end{align*}
And we (or rather \emph{Mathematica}) should be able to compute the latter, once we have figured out how the exponentials for the infinitesimal boost $u$, $v$, $w$ actually look like \ldots
}%

\newpage
\section{Integrals}

\begin{align*}
	A(a) &\ceq  \int_{\partial\varOmega} \lambda(x,a)^2 \, \dd S(x)
	\\
	V(a) &\ceq  \int_\varOmega  \lambda(x,a)^3 \, \dd x
\end{align*}
We would like to show that
\begin{align*}
	\frac{\dd}{\dd a} \frac{V(a)^2}{A(a)^3} > 0.
\end{align*}
This is equivalent to
\begin{align*}
	2 \, A(a) \, V'(a) - 3 \, A'(a) \, V(a) >0.
\end{align*}
\begin{align*}
	A'(a) &= \int_{\partial\varOmega}  2 \, \lambda(x,a) \, \pd_a\lambda(x,a) \, \dd S(x)
	\\
	V'(a) &= \int_\varOmega  3 \, \lambda(x,a)^2 \, \pd_a\lambda(x,a) \, \dd x
\end{align*}
\begin{align*}
	3 \, A'(a) \, V(a)
	&=
	6 \, \int_{\partial \varOmega} \int_\varOmega
		\lambda(x,a) \, \pd_a\lambda(x,a) \,
		\lambda(y,a)^3
	\, \dd y \, \dd S(x)
	\\
	2 \, A(a) \, V'(a)
	&=
	6 \, \int_\varOmega \int_{\partial\varOmega}
		\lambda(x,a)^2 \, \pd_a\lambda(x,a) \, \lambda(y,a)^2
	\, \dd S(y) \, \dd x
	\\
	&=
	6 \, \int_{\partial\varOmega} \, \int_\varOmega
		\lambda(x,a)^2 \, \pd_a\lambda(y,a) \, \lambda(y,a)^2
	\, \dd y \, \dd S(x)	
\end{align*}
Observe that 
\begin{align*}
	\partial ( \varOmega \times \varOmega) = \partial \varOmega \times \varOmega \cup \varOmega \times \partial \varOmega.
\end{align*}
So maybe we can find a nice $5$-form on $\varOmega$ such that
\begin{align*}
	3 \, A'(a) \, V(a)
	-
	2 \, A(a) \, V'(a)
	=
	\int_{\varOmega \times \varOmega} \, \dd \omega.
\end{align*}
and for which we can tell that the $\int_{\varOmega \times \varOmega} \, \dd \omega$ must be positive.

\begin{align*}
	\MoveEqLeft
	3 \, A'(a) \, V(a)
	-
	2 \, A(a) \, V'(a)
	\\
	&=
	6 \, 
	\int_M \int_M
	\Bigparen{
		\varphi(y,a)
		- \varphi(x,a)
	} \, \varphi(x,a) \, \varphi(y,a)^2
	\,  \pd_a\varphi(x,a)
	\, \dd \mu(y) \, \dd \mu(x)
	\\
	&=
	3 \, 
	\int_M \int_M
	\Bigparen{
		\varphi(y,a)
		- \varphi(x,a)
	} \, \varphi(x,a) \, \varphi(y,a)^2
	\,  \pd_a\varphi(x,a)
	\, \dd \mu(y) \, \dd \mu(x)
	+
	\\
	&\qquad
	3 \, 
	\int_M \int_M
	\Bigparen{
		\varphi(x,a)
		- \varphi(y,a)
	} \, \varphi(y,a) \, \varphi(x,a)^2
	\,  \pd_a\varphi(y,a)
	\, \dd \mu(y) \, \dd \mu(x)		
\end{align*}


%\begin{align*}
%	\frac{
%		\paren{\fint_M \varphi(x,a)^2 \, \vol }^3
%	}{
%		\paren{\fint_M \varphi(x,a)^3 \, \vol }^2
%	}
%\end{align*}
\begin{align*}
	\lambda(x)
	\ceq 
	\frac{1}{1+ 2 \, a\, x_1 + a^2 \, \nabs{x}^2}
\end{align*}
\begin{align*}
	\grad \lambda^2(x)
	=
	\frac{-4}{(1+ 2 \, a\, x_1 + a^2 \, \nabs{x}^2)^3}
	\,
	(a \, e_1 + a^2 \, x)
	=
	- 4 \, (a \, e_1 + a^2 \, x) \, \lambda^3.
\end{align*}
\begin{align*}
	\int_{\partial \varOmega} \lambda^2 \, \dd S(x)
	&=
	\int_{\partial \varOmega} \ninnerprod{N,\lambda^2 \,  X} \, \dd S(x)	
	\\
	&=
	\int_{\varOmega} \diver( \lambda^2 \,  X ) \, \dd x
	\\
	&=
	\int_{\varOmega}  \ninnerprod{ \grad \lambda^2 , X } \, \dd x	
	+
	\int_{\varOmega} \lambda^2 \,  \diver(X ) \, \dd x,
\end{align*}
where $X = r \frac{\partial}{\partial r}$.
Note that
\begin{align*}
	\ninnerprod {	\grad \lambda^2(x) , X}
	= - \lambda^3
	\paren{
	a \, r \, \cos(\varphi) \, \cos(\theta)
	+
	r \, (r + \sqrt{2}\, \cos(\theta))
	}
\end{align*}
Orthonormal basis
\begin{align*}
	e_1 \ceq\frac{\pd}{\pd r},\quad
	e_2 \ceq \frac{1}{\sqrt{2} + r \cos(\theta)}\frac{\pd}{\pd \varphi} \quad	
	e_3 \ceq\frac{1}{r}\frac{\pd}{\pd \theta}
	.
\end{align*}
\begin{align*}
	\diver(X)
	&=
	\innerprod{e_1, \frac{\partial}{\partial r} X}
	+
	\innerprod{e_2 , \frac{1}{\sqrt{2} + r \cos(\theta)} \frac{\partial}{\partial \varphi} X}	
	+
	\innerprod{e_3 , \frac{1}{r}\frac{\partial}{\partial \theta} X}		
	= 
	2
	+
	\frac{r\, \cos(\theta)}{\sqrt{2} + r \, \cos(\theta)}
\end{align*}

\newpage

\begin{align*}
	\int_{\partial \varOmega} \lambda(y,a)^2 \, \dd S(y)
	=
	\int_{\varOmega}  \ninnerprod{ \grad \lambda^2 (y,a) , X(y) } \, \dd y
	+
	\int_{\varOmega} \lambda^2(y,a) \,  \diver(X )(y) \, \dd y,
\end{align*}
\begin{align*}
	\frac{1}{2}\, A'(a) \, V(a)
	&=
	\int_\varOmega \int_{\partial \varOmega}
		\lambda(x,a)^3 \, \lambda(y,a) \, \pd_a\lambda(y,a)
	 \, \dd S(y) \, \dd x
	\\
	&=
	\int_{\varOmega} \int_\varOmega
		\lambda(x,a)^3 \,  \lambda(y,a) \, 
	\, \ninnerprod{ \pd_a \grad \lambda^2 (y,a) , X(y) }
	\, \dd y \, \dd x
	\\
	&\qquad 
	+
	\int_{\varOmega} \int_\varOmega
		\lambda(x,a)^3 \,  \lambda(y,a) \, \pd_a\lambda(y,a) 
	\,\diver(X)(y)
	\, \dd y \, \dd x
\end{align*}

\begin{align*}
	\frac{1}{3} \, A(a) \, V'(a)
	&=
	\int_\varOmega \int_{\partial\varOmega}
		\lambda(x,a)^2 \, \pd_a\lambda(x,a) \, \lambda(y,a)^2
	\, \dd S(y) \, \dd x
	\\
	&=
	\int_\varOmega \int_{\varOmega}
		\lambda(x,a)^2 \, \pd_a\lambda(x,a) \, \ninnerprod{ \grad \lambda^2 (y,a) , X(y) }
	\, \dd y \, \dd x		
	\\
	&\qquad
	+
	\int_\varOmega \int_{\varOmega}
		\lambda(x,a)^2 \, \pd_a\lambda(x,a) \, \lambda(y,a)^2
	\, \diver(X)(y)
	\, \dd y \, \dd x
\end{align*}
\begin{align*}
	\pd_a \grad \lambda^2
	=
	\grad (\pd_a\lambda^2)
	=
	2 \, \grad (\lambda \, \pd_a\lambda)
	=
	2 \, \lambda  \grad( \pd_a\lambda)
	+
	2 \, \pd_a\lambda  \grad \lambda
\end{align*}
\begin{align*}
	\lambda(x,a)^3 \,  \lambda(y,a) \, 
	\, \ninnerprod{ \pd_a \grad \lambda^2 (y,a) , X(y) }
\end{align*}
\begin{align*}
	\MoveEqLeft
	\frac{1}{2}\, A'(a) \, V(a) - \frac{1}{3} \, A(a) \, V'(a)
	\\
	&=
	\int_\varOmega \int_{\varOmega}
	\Bigparen{
		\lambda(x,a)^3 \, \lambda(y,a) \, \pd_a\lambda(y,a)
		-
		\lambda(x,a)^2 \, \pd_a\lambda(x,a) 
	}		
	\, \ninnerprod{ \grad \lambda^2 (y,a) , X(y) }
	\, \dd y \, \dd x		
	\\
	&\qquad
	+
	\int_\varOmega \int_{\varOmega}
	\Bigparen{	
		\lambda(x,a)^3 \,  \lambda(y,a) \, \pd_a\lambda(y,a) 
		-
		\lambda(x,a)^2 \, \pd_a\lambda(x,a) \, \lambda(y,a)^2
	}
	\, \diver(X)(y)
	\, \dd y \, \dd x
	\\
	&=
	\int_\varOmega \int_{\varOmega}
	\Bigparen{
		\lambda(x,a) \, \lambda(y,a) \, \pd_a\lambda(y,a)
		-
		\pd_a\lambda(x,a) 
	}		
	\, \lambda(x,a)^2 \,  \ninnerprod{ \grad \lambda^2 (y,a) , X(y) }
	\, \dd y \, \dd x		
	\\
	&\qquad
	+
	\int_\varOmega \int_{\varOmega}
	\Bigparen{	
		\lambda(x,a) \, \pd_a\lambda(y,a) 
		-
		\pd_a\lambda(x,a) \, \lambda(y,a)
	}
	\, \lambda(y,a) \, \lambda(x,a)^2
	\, \diver(X)(y)
	\, \dd y \, \dd x	
\end{align*}

	\newpage
\begin{align*}
	3 \int_\alpha^\beta A'(a) \, V(a) \, \dd a
	&=
	6  \int_\alpha^\beta\, \int_{\partial \varOmega} \int_\varOmega
		\lambda(x,a) \, \pd_a\lambda(x,a) \,
		\lambda(y,a)^3
	\, \dd y \, \dd S(x) \, \dd a
	\\
	&=
	3\brackets{
	\int_{\partial \varOmega} \int_\varOmega
		\lambda(x,a)^2
		\lambda(y,a)^3
	\, \dd y \, \dd S(x)
	}_{a= \alpha}^{a = \beta}
	\\
	&\qquad
	-
	9  \int_\alpha^\beta\, \int_{\partial \varOmega} \int_\varOmega
		\lambda(x,a)^2   \,
		\pd_a\lambda(y,a) \, \lambda(y,a)^2
	\, \dd y \, \dd S(x) \, \dd a	
	\\
	&=
	3\brackets{
	\int_{\partial \varOmega} \int_\varOmega
		\lambda(x,a)^2
		\lambda(y,a)^3
	\, \dd y \, \dd S(x)
	}_{a= \alpha}^{a = \beta}
	-
	3 
	\int_\alpha^\beta\, A(a) \, V'(a) \, \dd a		
\end{align*}

\begin{align*}
		2 \, A(a) \, V'(a)
	&=
	6 \, \int_\varOmega \int_{\partial\varOmega}
		\lambda(x,a)^2 \, \pd_a\lambda(x,a) \, \lambda(y,a)^2
	\, \dd S(y) \, \dd x
	\\
	&=
	6 \, \int_{\partial\varOmega} \, \int_\varOmega
		\lambda(x,a)^2 \, \pd_a\lambda(y,a) \, \lambda(y,a)^2
	\, \dd y \, \dd S(x)	
\end{align*}


\newpage

\section{Another take on the integrals}

Denote the Willmoretorus by $\varSigma(0) \subset \R^3$.
\begin{align*}
	\varSigma(x) &\ceq \iota( \iota(\varSigma(0) + (a,0,0)) 
	\\
	U_a(x) &\ceq \frac{\dd}{\dd a} \iota( \iota(x) + (a,0,0)).
\end{align*}

\begin{align*}
	A(a) &= \int_{\varSigma(a)} \, \dd \HausdorffMeasure^2(x)
	\\
	A'(a) &= -2 \int_{\varSigma(a)} H(x) \,  \ninnerprod{N(x), U_a(x) } \, \dd \HausdorffMeasure^2(x)	
	\\
	V(a) &= \frac{1}{3}
	\int_{\varSigma(a)}  \ninnerprod{N(x) , x} \, \dd \HausdorffMeasure^2(x)	
	\\
	V'(a) &=
	\int_{\varSigma(a)}  \ninnerprod{N(x) ,  U_a(x)} \, \dd \HausdorffMeasure^2(x)		
\end{align*}

\begin{align*}
	2 \, A(a) \, V'(a)	
	&=
	2
	\int_{\varSigma(a)} 	\int_{\varSigma(a)} 
%		\ninnerprod{N(x) ,  N(x)} \,
		 \ninnerprod{N(y) ,  U_a(y)}
	\, \dd \HausdorffMeasure^2(y)
	\, \dd \HausdorffMeasure^2(x)
	\\
	3 \, A'(a) \, V(a)
	&=
		- 2 \int_{\varSigma(a)} \int_{\varSigma(a)}
		H(x) \, \ninnerprod{N(x), U_a(x) } \,  \ninnerprod{N(y) , y} 
		\, \dd \HausdorffMeasure^2(y)	 \, \dd \HausdorffMeasure^2(x)	
\end{align*}

\begin{align*}
	\MoveEqLeft
	2 \, A(a) \, V'(a) - 3 \, A'(a) \, V(a)
	\\
	&=
	2
	\int_{\varSigma(a)}
	\bigparen{
		1 + 3 \, V(a) \, H(y)
	}
	\,
	\ninnerprod{N(y),  U_a(y)}
	\, \dd \HausdorffMeasure^2(y)
\end{align*}