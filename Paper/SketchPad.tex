\section{Questions}

\subsection{\Andy{I do not have any intuition about preservation ‘up to first order’.}}

\Henrik{%
Through a somewhat intransparent process (in the sense that I did not present their derivation in the \emph{Mathematica} notebook), I was able to determine the fundamental vector fields $\Funda[][\AmbSpace]$ for the three ``infinitesimal boosts'' $u$, $v$, $w \in \Lie(G)$, i.e. for three further generators that complement $\Lie(\SO(\AmbDim)) \oplus \Lie(\AmbSpace) \oplus \Lie(\R_{>0}) in \Lie(G)$, so that we obtain a full basis of $\Lie(G)$. 
In the notebook, I denoted these three fundamental vector fields by $U$, $V$, $W$, i.e., 
\begin{align*}
	U \ceq K^{\AmbSpace}_u,
	\quad
	V \ceq K^{\AmbSpace}_v,
	\qand
	W \ceq K^{\AmbSpace}_w.
\end{align*}
Thus by \eqref{eq:Funda}, we have 
\begin{align*}
	\Funda[u] \at_f =  U \circ f, 
	\quad
	\Funda[v] \at_f =  V \circ f, 
	\qand
	\Funda[w] \at_f =  W \circ f.
\end{align*}

We can now study area $\Area \colon \ConfSpace \to \R$ and enclosed volume $\Volume \colon \ConfSpace \to \R$ along an orbit $\Orbit \ceq (G f)$ of $G$. 
By the chain rule we have for each differentiable curve 
$g \colon ]-\epsilon,\epsilon[ \to G$ with $g(0) = 1$ and $g'(0) \in \Lie(G)$ 
that
\begin{align*}
		\frac{\dd}{\dd t} \Area( L_{g(t)}(f) ) \At_{t=0} = D \Area(f) \,(\Funda[\xi] \at_f)
\end{align*}
and analogously
\begin{align*}
		\frac{\dd}{\dd t} \Volume( L_{g(t)}(f) ) \At_{t=0} = D \Volume(f) \,(\Funda[\xi] \at_f).
\end{align*}

And what I did in the notebook is just checking for each torus $f$ of revolution with circular contour (TRCC for short) that

\begin{align*}
	D \Area(f) \,(\Funda[u] \at_f) &= 0.
	\\
	D \Area(f) \,(\Funda[v] \at_f) &= 0,
	\\
	D \Area(f) \,(\Funda[w] \at_f) &= 0,
	\\
	D \Volume(f) \,(\Funda[u] \at_f) &= 0,
	\\
	D \Volume(f) \,(\Funda[v] \at_f) &= 0,
	\\
	D \Volume(f) \,(\Funda[w] \at_f) &= 0.
\end{align*}

Of course, this does not necessarily imply that $\Area$ and $\Volume$ are constant on the whole orbit $\mathcal{O} = (G f)$, not even on an arbitrarily small neighborhood of $f$. While the boost $\exp(t \, w)$ ``in $z$-direction'' (along the axis of revolution) sends a TRCC to a TRCC (the contour has to be circular again and boosting along the axis of revolution preserves rotation symmetry; the parameterization might change considerably, but we are not interested in the parameterization, right?), the boosts $\exp(t \, u)$ and $\exp(t \, v )$ along the $x$ and $y$ direction should deform TRCC into Dupin cyclides. And so I do not expect that $\exp(t \,u)$, $\exp(t \,v)$ preserve both area and volume. But I could be wrong.
}%

\subsection{\Andy{With your symbolic setup in \emph{Mathematica} Henrik, is it easy to look at the 2nd order change? Maybe we will even see what makes the Clifford torus isoperimetric ratio so special?}}

\Henrik{%
Could be worth a try. Computing the Hessian at a critical point is typically not that difficult because it actually does not depend on the employed Riemannian metric and its connection:

Denote the orbit of $f$ under $G$ by $\Orbit = L_G(f)$. The fundamental vector fields $\Funda$ span the tangent space  $T_f  \Orbit$. If $\Function \colon \Orbit \to \R$ is twice differentiable, then for any Riemannian metric $h$ on $\Orbit$ we have
\begin{align*}
 	\Hess^{h}(\Function)( \Funda[\xi] , \Funda[\eta]) 
 	= 
 	\bigparen{ \Funda[\xi] \Funda[\eta] \Function } \at_f 
 	- 
 	\dd \Function(f) \, \nabla^h_{\Funda[\xi]} \Funda[\eta],
 	\quad
 	\text{for all $\xi$, $\eta \in G$,}
\end{align*} 	
where $\nabla^h$ is the Levi-Civita connection of $h$.
So in general, the result depends on the choice of a metric! But if $\dd \Function \at f = 0$ (like in the cases $\Function = \Area |_{\Orbit}$ and $\Function = \Volume |_{\Orbit}$ for example), then the metric-dependent term just vanishes and we get
\begin{align*}
 	\Hess^{h}(\Function)( \Funda[\xi] , \Funda[\eta]) 
 	= 
 	\bigparen{ \Funda[\xi] \Funda[\eta] \Function } \at_f 
	.\textsl{•}
\end{align*}
And we (or rather \emph{Mathematica}) should be able to compute the latter, once we have figured out how the exponentials for the infinitesimal boost $u$, $v$, $w$ actually look like \ldots
}%